% Options for packages loaded elsewhere
\PassOptionsToPackage{unicode}{hyperref}
\PassOptionsToPackage{hyphens}{url}
%
\documentclass[
  12pt,
]{article}
\usepackage{amsmath,amssymb}
\usepackage{lmodern}
\usepackage{ifxetex,ifluatex}
\ifnum 0\ifxetex 1\fi\ifluatex 1\fi=0 % if pdftex
  \usepackage[T1]{fontenc}
  \usepackage[utf8]{inputenc}
  \usepackage{textcomp} % provide euro and other symbols
\else % if luatex or xetex
  \usepackage{unicode-math}
  \defaultfontfeatures{Scale=MatchLowercase}
  \defaultfontfeatures[\rmfamily]{Ligatures=TeX,Scale=1}
\fi
% Use upquote if available, for straight quotes in verbatim environments
\IfFileExists{upquote.sty}{\usepackage{upquote}}{}
\IfFileExists{microtype.sty}{% use microtype if available
  \usepackage[]{microtype}
  \UseMicrotypeSet[protrusion]{basicmath} % disable protrusion for tt fonts
}{}
\makeatletter
\@ifundefined{KOMAClassName}{% if non-KOMA class
  \IfFileExists{parskip.sty}{%
    \usepackage{parskip}
  }{% else
    \setlength{\parindent}{0pt}
    \setlength{\parskip}{6pt plus 2pt minus 1pt}}
}{% if KOMA class
  \KOMAoptions{parskip=half}}
\makeatother
\usepackage{xcolor}
\IfFileExists{xurl.sty}{\usepackage{xurl}}{} % add URL line breaks if available
\IfFileExists{bookmark.sty}{\usepackage{bookmark}}{\usepackage{hyperref}}
\hypersetup{
  pdftitle={A Right Restricted},
  pdfauthor={Kevin Morris},
  hidelinks,
  pdfcreator={LaTeX via pandoc}}
\urlstyle{same} % disable monospaced font for URLs
\usepackage[margin=1in]{geometry}
\usepackage{longtable,booktabs,array}
\usepackage{calc} % for calculating minipage widths
% Correct order of tables after \paragraph or \subparagraph
\usepackage{etoolbox}
\makeatletter
\patchcmd\longtable{\par}{\if@noskipsec\mbox{}\fi\par}{}{}
\makeatother
% Allow footnotes in longtable head/foot
\IfFileExists{footnotehyper.sty}{\usepackage{footnotehyper}}{\usepackage{footnote}}
\makesavenoteenv{longtable}
\usepackage{graphicx}
\makeatletter
\def\maxwidth{\ifdim\Gin@nat@width>\linewidth\linewidth\else\Gin@nat@width\fi}
\def\maxheight{\ifdim\Gin@nat@height>\textheight\textheight\else\Gin@nat@height\fi}
\makeatother
% Scale images if necessary, so that they will not overflow the page
% margins by default, and it is still possible to overwrite the defaults
% using explicit options in \includegraphics[width, height, ...]{}
\setkeys{Gin}{width=\maxwidth,height=\maxheight,keepaspectratio}
% Set default figure placement to htbp
\makeatletter
\def\fps@figure{htbp}
\makeatother
\setlength{\emergencystretch}{3em} % prevent overfull lines
\providecommand{\tightlist}{%
  \setlength{\itemsep}{0pt}\setlength{\parskip}{0pt}}
\setcounter{secnumdepth}{5}
\usepackage{rotating}
\usepackage{setspace}
\usepackage{lscape}
\usepackage{pdfpages}
\newcommand{\beginsupplement}{\setcounter{table}{0}  \renewcommand{\thetable}{A\arabic{table}} \setcounter{figure}{0} \renewcommand{\thefigure}{A\arabic{figure}}}
\usepackage{lineno}
\usepackage{booktabs}
\usepackage{longtable}
\usepackage{array}
\usepackage{multirow}
\usepackage{wrapfig}
\usepackage{float}
\usepackage{colortbl}
\usepackage{pdflscape}
\usepackage{tabu}
\usepackage{threeparttable}
\usepackage{threeparttablex}
\usepackage[normalem]{ulem}
\usepackage{makecell}
\usepackage{xcolor}
\ifluatex
  \usepackage{selnolig}  % disable illegal ligatures
\fi
\newlength{\cslhangindent}
\setlength{\cslhangindent}{1.5em}
\newlength{\csllabelwidth}
\setlength{\csllabelwidth}{3em}
\newenvironment{CSLReferences}[2] % #1 hanging-ident, #2 entry spacing
 {% don't indent paragraphs
  \setlength{\parindent}{0pt}
  % turn on hanging indent if param 1 is 1
  \ifodd #1 \everypar{\setlength{\hangindent}{\cslhangindent}}\ignorespaces\fi
  % set entry spacing
  \ifnum #2 > 0
  \setlength{\parskip}{#2\baselineskip}
  \fi
 }%
 {}
\usepackage{calc}
\newcommand{\CSLBlock}[1]{#1\hfill\break}
\newcommand{\CSLLeftMargin}[1]{\parbox[t]{\csllabelwidth}{#1}}
\newcommand{\CSLRightInline}[1]{\parbox[t]{\linewidth - \csllabelwidth}{#1}\break}
\newcommand{\CSLIndent}[1]{\hspace{\cslhangindent}#1}

\title{A Right Restricted\thanks{TKTKT}}
\usepackage{etoolbox}
\makeatletter
\providecommand{\subtitle}[1]{% add subtitle to \maketitle
  \apptocmd{\@title}{\par {\large #1 \par}}{}{}
}
\makeatother
\subtitle{Understanding the Introduction and Passage of Restrictive Voting Laws}
\author{Kevin Morris\footnote{Researcher, Brennan Center for Justice (kevin.morris{[}at{]}nyu.edu).}}
\date{March 23, 2022}

\begin{document}
\maketitle
\begin{abstract}
TKTKT
\end{abstract}

\pagenumbering{gobble}
\pagebreak

\pagenumbering{arabic}
\doublespacing

\hypertarget{introduction}{%
\section{Introduction}\label{introduction}}

On May 7, 2021, Texas legislators in the state's House of Representatives debated and passed Senate Bill 7, an omnibus bill restricting voting in various ways. The bill would reduce access to mail voting, ban drive-through and 24-hour voting, and require large counties to redistribute their polling places away from Black and Latino neighborhoods (\protect\hyperlink{ref-Ura2021a}{Ura 2021a}; \protect\hyperlink{ref-Ura2021}{Ura, Essig, and Dong 2021}). Although this particular bill would ultimately fail after Democratic legislators broke quorum before the final vote by leaving the state, many of these provisions would ultimately become law as part of Senate Bill 1 during a special session called by the governor (\protect\hyperlink{ref-Ura2021b}{Ura 2021b}).

The debate in the House, however, was marked by an argument about a single phrase, used in the opening text of the bill. Senate Bill 7's self-described purpose was to ``detect and punish fraud and preserve the purity of the ballot box'' (§1.02). This phrase---``the purity of the ballot box''---has a long history in Texas, enshrined in the state's 1876 Constitution and used to defend the state's white primary that effectively shut nonwhites out of the political process for decades (\protect\hyperlink{ref-Knowles2021}{Knowles 2021}; \protect\hyperlink{ref-Morris2021b}{Morris and Pérez 2021}). Democratic representative Rafael Anchía questioned the bill author's use of this ``specific set of words that has a lot of meaning in state history,'' (quoted in \protect\hyperlink{ref-Knowles2021}{Knowles 2021}) saying the constitutional provision ``was drafted specifically to disenfranchise Black people.'' The implication was clear: Texas legislators in 2021 were tapping into long-standing legal racism to pass new legislation that would disproportionately impact voters of color. The phrase was dropped from the final version of Senate Bill 1, passed in August.

The twin features of the introduction to Texas' Senate Bill 7---protection against fraud and appeals to purity---typified the 2021 legislative session around the country. After losing his re-election bid in November, then-president Donald Trump claimed repeatedly that the election had been stolen (\protect\hyperlink{ref-Dale2020}{Dale 2020}), a claim he has continued to maintain and that some 70\% of registered Republicans believed by early 2022 (\protect\hyperlink{ref-Cuthbert2022}{Cuthbert and Theodoridis 2022}). Many state legislators also justified their support for restrictive legislation in terms similar to Oklahoma State Representative Sean Roberts (sponsor of the restrictive restrictive HB 2842 and HB 2847), who told reporters that ``{[}I{]}t was very clear that the election was stolen from President Trump. We must do everything we can to close those loopholes'' (quoted in \protect\hyperlink{ref-May2022}{May 2022}). Concerns about election security were not limited to state legislators: 147 Congressional Republicans voted against the certification of the 2020 presidential election (\protect\hyperlink{ref-Yourish2021}{Yourish, Buchanan, and Lu 2021}).

Despite these widespread beliefs, no evidence of fraud arose following the 2020 election, calling into question the veracity of these stated reasons for supporting the restrictive legislation. As the New York Times explained: ``After bringing some 60 lawsuits, and even offering financial incentive for information about fraud, Mr.~Trump and his allies have failed to prove definitively any case of illegal voting on behalf of their opponent in court---not a single case of an undocumented immigrant casting a ballot, a citizen double voting, nor any credible evidence that legions of the voting dead gave Mr.~Biden a victory that wasn't his'' (\protect\hyperlink{ref-Rutenberg2020}{Rutenberg, Corasaniti, and Feuer 2020}). The lack of evidence of fraud was used to insinuate that this restrictive legislation is driven by racial animus (e.g., \protect\hyperlink{ref-BaconJr.2022}{Bacon 2022}).

This project relies on a comprehensive survey of voting-related bills introduced around the country in 2021 systematically collected by the Brennan Center for Justice to adjudicate between these competing explanations for the introduction and passage of restrictive voting legislation.\footnote{See \url{https://www.brennancenter.org/our-work/research-reports/voting-laws-roundup-december-2021}. Provision-level data available from the Brennan Center upon request.} I start by examining these bills in light of already-existing voting regimes. If concerns about fraudulent voting have purchase, I expect to find more restrictive bills introduced and passed in states with more permissive voting laws---and therefore, perhaps, greater opportunity for fraud.

I also consider whether these restrictive laws can be explained by partisanship. Republicans may be passing bills expected to benefit their electoral prospects, as some have argued (e.g., \protect\hyperlink{ref-Kennedy2021}{Kennedy and Anderson 2021}). A partisan explanation would find more restrictive provisions in competitive states controlled by Republicans looking to cement a perhaps-tenuous advantage (\protect\hyperlink{ref-Hicks2015}{Hicks et al. 2015}).

Finally, I interrogate whether legislative behavior related to voting bills can be explained by white backlash and racial threat as many observers have claimed. If these restrictive laws are a response to rising non-white political power, I expect to find their introduction and passage concentrated in states with large non-white populations, and support for them concentrated in the whitest legislative districts in racially diverse states (\protect\hyperlink{ref-Andrews2015}{Andrews and Seguin 2015}).

The results are unequivocal: states with large nonwhite populations where voting was \emph{already} restricted made it even more difficult to vote in 2021. Restrictive legislation was not introduced or passed more in states with more lax voting regimes, and President Biden's 2020 voteshare was not related to state-level legislative behavior. Moreover, it was representatives from the whitest districts in the least-white states that were the most likely to sponsor restrictive legislation. While partisanship does explain legislative sponsorship---representatives of districts with higher Trump voteshare were the most likely to sponsor restrictive legislation---these strong racial patterns remain after accounting for polarized voting. In short, theories of white backlash and racial threat explain the introduction and passage of regressive voting laws in 2021 far better than do theories of partisanship or fears of fraud founded in actual voting laws.

\hypertarget{recent-work-on-restrictive-voting-laws}{%
\section{Recent Work on Restrictive voting Laws}\label{recent-work-on-restrictive-voting-laws}}

Over the past 15 years, scholars have explored the introduction and passage of restrictive of restrictive voting laws across the country. This work has largely focused on state-level factors, with a general consensus that these laws find the most fertile ground in states with large demographic change and a growing nonwhite electorate (\protect\hyperlink{ref-Bentele2013}{Bentele and O'Brien 2013}), where large numbers of Black Americans reside (\protect\hyperlink{ref-Behrens2003}{Behrens, Uggen, and Manza 2003}), and in electorally competitive states where Republicans hold a slight edge (\protect\hyperlink{ref-Hicks2015}{Hicks et al. 2015}).

\protect\hyperlink{ref-Behrens2003}{Behrens, Uggen, and Manza} (\protect\hyperlink{ref-Behrens2003}{2003}) uses a historical approach to understand the passage of laws disenfranchising citizens convicted of felony offenses. As they note, all but two American states restrict voting rights for at least some incarcerated citizens; the two that do not---Maine and Vermont---are also the two whitest states in the nation. Behrens and colleagues document the rise of these restrictive laws in the aftermath of the passage of the 14th and 15th Amendments, expanding formal citizenship and granting voting rights to Black men. Drawing on \protect\hyperlink{ref-Blumer1958}{Blumer} (\protect\hyperlink{ref-Blumer1958}{1958}) and other scholars of group threat, they argue that white (male) Americans were threatened by the prospect that their sole control over the political domain was no longer so secure. Of course, their claims to racial political dominance were threatened proportionate to the number of nonwhite potential voters; as such, states with larger nonwhite populations had political incentives to develop new ways to disenfranchise Black men. They find strong support for the theory that the widespread adoption of felony disenfranchisement rules rose from this threat. ``Our key finding can be summarized concisely and forcefully,'' they write. ``The racial composition of state prisons is firmly associated with the adoption of state felon disenfranchisement laws. States with greater nonwhite prison populations have been more likely to ban convicted felons from voting than states with proportionally fewer nonwhites in the criminal justice system'' (\protect\hyperlink{ref-Behrens2003}{Behrens, Uggen, and Manza 2003, 596}). Their conclusions have been corroborated more recently. \protect\hyperlink{ref-Eubank2022}{Eubank and Fresh} (\protect\hyperlink{ref-Eubank2022}{2022}) finds that states subject to strict federal oversight under the 1965 Voting Rights Act's Section 5 selectively increased the incarceration of Black Americans, providing further evidence that increased political opportunity for racial minorities leads white majorities to seek other ways of restricting their effective power.

Of course, the incarceration of citizens and subsequent legal disenfranchisement is perhaps only the most drastic example of curtailing access to the ballot.\footnote{It bears noting, however, that being drastic does not mean it is uncommon. More than 6\% of Black Americans were legally disenfranchised in 2020 due to a felony conviction. This number topped 10\% in 7 of the 33 states where the Black voting age population exceeded 100,000 (\protect\hyperlink{ref-Uggen2020}{Uggen et al. 2020}).} Might less extreme attempts to limit the pool of eligible voters follow a similar pattern? And do such considerations structure legislative behavior in the modern era? \protect\hyperlink{ref-Bentele2013}{Bentele and O'Brien} (\protect\hyperlink{ref-Bentele2013}{2013}) consider the introduction and passage of 5 types of restrictive voting legislation (``photo identification requirements, proof of citizenship requirements, laws that introduce restrictions on voter registration, restrictions on absentee and early voting, and restrictions on participation by felons'' (1095)) over the 2006--2011 period. They conclude that the strongest predictor of the introduction and passage of restrictive voting laws is the political power demonstrated by racial and ethnic minorities, arguing that ``legislative developments in this policy area remain heavily shaped by racial considerations'' (\protect\hyperlink{ref-Bentele2013}{Bentele and O'Brien 2013, 1089}). At the same time, they find no evidence that prevalence of voter fraud impacted the introduction of restrictive provisions and that it is ``only a minor contributing factor'' to the passage of these laws in 2011 (1103).

A further insight from \protect\hyperlink{ref-Bentele2013}{Bentele and O'Brien} (\protect\hyperlink{ref-Bentele2013}{2013})---that restrictive provisions are passed most frequently in electorally-competitive states---is corroborated by \protect\hyperlink{ref-Hicks2015}{Hicks et al.} (\protect\hyperlink{ref-Hicks2015}{2015}). Looking specifically at the introduction and passage of restrictive voter identification laws in the early 2000s. They find that states with more Republican legislators were considerably more likely to enact these provisions---but that ``Republicans have not pursued this scorched-earth policy in all states, nor have they done so consistently over time'' (29--30). Instead, Republicans were more likely to pass these bills where their electoral majorities were slim. Hicks et al. (\protect\hyperlink{ref-Hicks2015}{2015, 18}) thus conclude that ``where elections are competitive, the furtherance of restrictive voter ID laws is a means of maintaining Republican support while curtailing Democratic electoral gains.'' Other work (e.g., \protect\hyperlink{ref-Biggers2015}{Biggers and Hanmer 2015}; \protect\hyperlink{ref-Wang2012}{Wang 2012}) also indicates that restrictive voting laws are passed by Republican-dominated legislatures to shore up flagging electoral majorities.

This scholarship sheds important light on where restrictive voting laws are the most likely to go into effect, and the results are not encouraging. There is strong evidence that racial threat predicts the passage of these restrictive bills across the country, even as legislators proclaim that the changes are needed to combat widespread fraud (see, for instance, \protect\hyperlink{ref-Piven2009}{Piven, Minnite, and Groarke 2009}; \protect\hyperlink{ref-Minnite2010}{Minnite 2010}). Important as this research has been, however, it fails to explain the full set of dynamics between demographic composition and bill introduction. The explosion in the introduction of restrictive voting laws in 2021 makes this clear: according to the data from the Brennan Center for Justice used throughout this project, just one state (Vermont) introduced \emph{no} voting bills in 2021 containing no restrictive provisions. Moreover, the number of restrictive provisions introduced and passed in 2021 has little historical precedent: 880 restrictive provisions were introduced and 93 were passed. By way of comparison, \protect\hyperlink{ref-Bentele2013}{Bentele and O'Brien} (\protect\hyperlink{ref-Bentele2013}{2013})---which also used data from the Brennan Center---calls roughly the roughly 20 passed provisions in 2011 a ``dramatic increase'' (1088; see their Figure 2).

Clearly, something more complex than state-level factors are at play in the contemporary push to restrict voting rights. By considering not only state-level factors but also examining the demographics of the districts represented by legislators who introduce, co-sponsor, and vote for these restrictive bills, this project marks a significant step forward in understanding how racial threat's influence on the policy-making process is mediated by factors at multiple political levels. The following section steps back to engage with the (racial) threat literature and, more specifically, consider how spatially-situated theories of threat help us to formulate expectations about the roles played by state and local factors in the introduction and passage of restrictive voting laws.

\hypertarget{a-changing-electorate-and-threat}{%
\section{A Changing Electorate and Threat}\label{a-changing-electorate-and-threat}}

Scholars across the social sciences have long noted the importance of threat to the policy-making process; indeed, each of the studies discussed in the previous section implicitly or explicitly draw on this literature. \protect\hyperlink{ref-Tilly1978}{Tilly} (\protect\hyperlink{ref-Tilly1978}{1978}) separates collective action into three categories: defensive, offensive, and preparatory (73). Of these, two---defensive and preparatory---are explicitly linked to threats, where political actors pool their resources to fend of challenges to their interests, or to regain what has already been lost. \protect\hyperlink{ref-Beck2000}{Beck} (\protect\hyperlink{ref-Beck2000}{2000}) extends this theory to note that defensive actors need only \emph{perceive} that their interests have been compromised to mobilize in a reactionary way; the \emph{reality} of any worsened station is perhaps less important. These threats can take multiple forms, be they economic, political, or demographic (\protect\hyperlink{ref-VanDyke2002}{Van Dyke and Soule 2002}).

In recent years, increasing attention has been paid to how different levels of spatial organization and threat can interact with one another (\protect\hyperlink{ref-Zhang2018}{Zhang and Zhao 2018}). \protect\hyperlink{ref-Tilly2015}{Tilly and Tarrow} (\protect\hyperlink{ref-Tilly2015}{2015}) explains how social movements can undergo what they call an ``upward scale shift,'' which they say ``moves contention beyond its local origins, touches on the interests and values of new actors, involves a shift of venue to sites where contention may be more or less successful, and can threaten other actors or entire regimes'' (125). In other words, political actors may move beyond the local context to make use of institutional tools available only at higher levels of government.

Although scholars have recognized the importance of space and scale in political and legislative activity, relatively less attention has been paid to how the \emph{interaction} of conditions at these different levels plays out. A notable exception to this is \protect\hyperlink{ref-Andrews2015}{Andrews and Seguin} (\protect\hyperlink{ref-Andrews2015}{2015}), which explores how racial threat, group contact, and differential levels of government structure legislative activity. They argue that ``threat arises primarily from interactions between spatially proximate units at the local level\ldots{} and therefore higher-level policy change at the state level is not reducible to the variables driving local policy'' (476). In other words, examining local and state characteristics alone is not sufficient to understand legislator support for racially conservative policy changes; instead, responses to racial threat arise from the \emph{interaction} of these circumstances.

Threat is clearly a major driver of policy in the United States. As the research on voting laws makes clear, states pass more restrictive legislation when there are more racial minorities whose political power threatens the established power structure. However, sociological models of mobilization, scale shifts, and geographical interplay pushes us to think more seriously about precisely where the support for racially-restrictive legislation comes from. On the one hand, we might expect that racially-diverse localities where whites maintain a political edge would feel the most threatened by nonwhite voters, leading their representatives to support more restrictive legislation. On the other hand, white and homogeneous areas of diverse states might be threatened by rising political power elsewhere in their own states, and their homogeneity might provide the basis for coordinated pressure on their representatives. Identifying the source of legislative support for racially-restrictive policymaking \emph{within} states is of key importance for better understanding the geographical / political topography of racial threat.

\hypertarget{methods-and-expectations}{%
\section{Methods and Expectations}\label{methods-and-expectations}}

Throughout our analyses, I rely on the Voting Laws Roundup, a project of the Brennan Center for Justice at NYU School of Law. The Brennan Center systematically reviews all laws introduced around the country that relate to voting and the administration of elections in each state. The Brennan Center identifies these bills using string-searches in Westlaw, and then separates each bill introduced into its constituent provisions, using two coders to designate each provision as ``restrictive,'' ``neutral,'' or ``expansive.'' Each provision is also assigned to a category describing its effect (categories include effects such as ``voter ID,'' ``polling place count,'' or ``funding for poll workers''). According to the Brennan Center, a bill's provisions are identified when a bill first includes provisions related to voting, and updated if a bill is passed. In other words, if a bill is introduced with some voting provisions, is subsequently amended to include other voting provisions, but ultimately fails to pass, only the original provisions are included. Figure \ref{fig:cols} shows the categorical breakdown of restrictive provisions introduced and passed, while Figure \ref{fig:maps} shows the geographical distribution of these provisions.

\begin{figure}[!ht]

{\centering \includegraphics{bad_bills_files/figure-latex/cols-1} 

}

\caption{\label{fig:cols}Categories of Restrictive Provisions, 2021}\label{fig:cols}
\end{figure}

\begin{figure}[!ht]

{\centering \includegraphics{bad_bills_files/figure-latex/maps-1} 

}

\caption{\label{fig:maps}Restrictive Provisions, 2021}\label{fig:maps}
\end{figure}

These provisions are then merged with data from LegiScan, which tracks bills of all kinds in statehouses throughout the United States. The LegiScan data includes information on bill sponsorship, roll-call votes, and dates of introduction. These data are used to identify the (co)sponsors of all the voting laws identified by the Brennan Center, as well as the legislators voting in favor or against them.

To account for pre-2021 variation in the electoral landscape, I incorporate each state's score on the (pre-COVID) 2020 Cost of Voting Index (\protect\hyperlink{ref-Schraufnagel2020}{Schraufnagel, Pomante II, and Li 2020}) (COVI). This index captures many of the same items tracked by the Brennan Center, such as whether a state has same-day registration, voter ID laws, and the number of days of early voting. The index is widely used by social scientists to measure the difficulty of voting (e.g. \protect\hyperlink{ref-Juelich2020}{Juelich and Coll 2020}; \protect\hyperlink{ref-Pabayo2021}{Pabayo et al. 2021}; \protect\hyperlink{ref-Rackey2022}{Rackey and Camarillo 2022}). A higher value of the COVI indicates that voting is more difficult in that state, and the 2020 COVI ranges from -2.9 in Oregon to 1.4 in New Hampshire.

I also control for the partisan control of each state in two ways. Following \protect\hyperlink{ref-Hicks2015}{Hicks et al.} (\protect\hyperlink{ref-Hicks2015}{2015}), it seems possible that electorally competitive states where Republicans hold unified power would be most likely to introduce and pass bad provisions. I thus include 2 dummies: one measuring whether the state was competitive (that is, Biden received between 45\% and 55\% of votes), and one measuring whether Republicans held unified control in 2021. Data on electoral competitiveness comes from the MIT Election Data and Science Lab (\protect\hyperlink{ref-MITElectionDataandScienceLab2021}{2021}), and data on partisan control comes from the National Conference of State Legislatures\footnote{See \url{https://www.ncsl.org/documents/elections/Legis_Control_2-2021.pdf}.} Although Nebraska's unicameral state legislature is formally nonpartisan, they are considered to be under unified Republican control for the purposes of this study. I also include this chamber in the upper-chamber analyses.

The legislative district-level analyses also include estimates of Joe Biden's voteshare in the 2020 presidential election. I estimate each Biden's voteshare in each district by aggregating up from precinct-level results published by the Voting Election and Science Team (\protect\hyperlink{ref-VotingandElectionScienceTeam2022}{2022}). I assign each precinct to the upper- and lower-chamber district in which its geographical center is located.\footnote{While this will not perfectly estimate voteshare in chambers where precincts cross district lines, there is little reason to expect this will systematically distort voteshare estimates.} This coverage is not perfect (Kentucky and West Virginia do not publish data at the precinct-level and therefore are excluded), but more than 96\% of Americans live in chambers for which presidential voteshare can be estimated using this methodology.

In addition to these measures of voting legislation and policy environment, I incorporate demographic data at the state and legislative-district level from the Census Bureau's American Communities Survey. These estimates are the 5-year numbers ending with 2020.

Based on the literature discussed above, I pose competing hypotheses at the state level:

\begin{itemize}
\item
  \textbf{H1a}: Restrictive voting legislation is best explained by theories of partisan advantage. Therefore, other things being equal, electorally-competitive states with unified Republican control were the most likely to introduce and pass restrictive provisions.
\item
  \textbf{H1b}: Restrictive voting legislation is best explained by theories of racial threat and white backlash. Therefore, other things being equal, states where whites make up a smaller share of the population were more likely to introduce and pass restrictive voting provisions. I expect this to be especially true in states with unified Republican control---that is, in more racially diverse states where the conservative party can pass restrictive bills into law.
\end{itemize}

After presenting tests of these hypotheses at the state-level, I move to analyses at the level of the legislative chamber. The empirical framework is the same, though the dependent variable changes from the \emph{count} of restrictive provisions to a dummy variable indicating \emph{whether} a legislator sponsored restrictive provisions, due to smaller scale.

In light of recent sociological work on the geography of racial threat (\protect\hyperlink{ref-Andrews2015}{Andrews and Seguin 2015}), \textbf{H2}: I expect that whiter districts in less-white states were the most likely to be represented by legislators that sponsored restrictive voting provisions.

\hypertarget{results}{%
\section{Results}\label{results}}

\hypertarget{state-level-legislative-behavior}{%
\subsection{State-Level Legislative Behavior}\label{state-level-legislative-behavior}}

In Table \ref{tab:state}, I present the results of tests on my first set of hypotheses. In Models 1--3, the dependent variable is the count of restrictive provisions introduced; in Models 4--6, it is the number of restrictive provisions passed. In each case, I use a robust regression (using \texttt{rlm()} in R from the \texttt{MASS} package) to prevent outlier states like Texas from exerting too much influence on the models.

\newpage
\begin{landscape}
\input{"../temp/state_reg.tex"}
\end{landscape}
\newpage

Because interaction coefficients can be difficult to interpret in table format, I here present the predicted probabilities of counts of introduced and passed provisions in Figure \ref{fig:mefs-state}. These results are presented with and without covariates to visually demonstrate how their inclusion alters the substantive relationships of interest. The top panels plot the results from Models 1 and 3 from Table \ref{tab:state}; the bottom panels present Models 4 and 6. In Panels (C) and (D), other covariates are held at their means.

\begin{figure}[!ht]

{\centering \includegraphics{bad_bills_files/figure-latex/mefs-state-1} 

}

\caption{\label{fig:mefs-state}Restrictive Provisions Introduced and Passed, by Race and 2020 COVI}\label{fig:mefs-state}
\end{figure}

We can easily see from Table \ref{tab:state} and Figure \ref{fig:mefs-state} that the wave of restrictive provisions in 2021---both those introduced and those passed---have much to do with partisanship \emph{and} race. States that were not under unified Republican control\footnote{Although Nebraska's unicameral legislature is formally nonpartisan, I categorize it as a state with unified Republican control.} neither introduced nor passed restrictive voting provisions in appreciable numbers. Among states with unified Republican control, however, race appears to be a driving factor: even after controlling for competitiveness in the states, less-white Republican-dominated states introduced and passed far more restrictive provisions than whiter states with unified Republican control. This is perhaps unsurprising: Texas, Georgia, Florida, and Arizona are all fully controlled by the Republican party, and each of these states passed either an omnibus bill or a series of restrictive laws in 2020. Moreover, as a comparison of Models 1 and 3, and 4 and 6 indicate, the relationships between state control, race, and restrictive provisions is not highly influenced by the inclusion of additional covariates. The coefficients on the independent variables of interest do not move very much.

Table \ref{tab:state} also provides some insight into the relationship between competitiveness, Republican domination, and restrictive provisions. In the naive models (2 and 5), few of the coefficients are significant at the 95\% confidence level. These models indicate that perhaps competitive states were more likely to introduce more restrictive provisions, and that competitive states with unified Republican control were perhaps more likely to pass restrictive provisions than other competitive states. After accounting for the other covariates in the models---including racial characteristics interacted with Republican control---, it seems that more restrictive provisions were introduced (but not passed) in states with unified Republican control. Competitive states with unified Republican control passed more restrictive provisions, ceteris paribus, though the negative (statistically non-significant) coefficient on competitiveness in other states reduces the substantive importance of this relationship.

To be sure, regression models with just 50 observations are necessarily limited. Nevertheless, these results indicate that there are substantive relationships between race and restrictive voting provisions in states with unified Republican control. Table \ref{tab:state} also indicates that competitive Republican-controlled states passed more restrictive provisions than other competitive states, but the relationship between competitiveness, partisan control, and restrictive voting laws is far more tenuous than those of race and partisanship. In short: theories of racial threat and white backlash do a far better job of explaining the observed patterns in the 2021 legislative context than do arguments about shoring up overly-permissive voting regimes, or those pointing only to partisan explanations.

\hypertarget{district-level-legislative-behavior}{%
\subsection{District-Level Legislative Behavior}\label{district-level-legislative-behavior}}

Having determined that the geographic concentration of the restrictive voting bills in 2021 was highly consistent with theories of racial threat, I turn to the sub-state level to better understand these processes. As discussed above, I expect to find evidence of backlash against racial threat in the whitest parts of the least white states.

Table \ref{tab:chamb} presents the results of OLS regressions investigating the likelihood that a legislator from a given district sponsored at least one restrictive voting provision in 2021.\footnote{The ACS and precinct-level results data are missing in a handful of small legislative districts, explaining the variability in the observation count. However, the share of the population living in upper and lower chambers without demographic data is just 0.5\% and 1.1\%, respectively.} About 22\% of lower chamber legislators and 20\% of upper chamber legislators sponsored restrictive provisions.

\input{"../temp/district_reg.tex"}

As before, I plot the predicted probability of bill sponsorship for provisions to aid in the interpretation of the substantive relationships. Figure \ref{fig:mef-dis} presents like predicted likelihood of the sponsorship of a provision at different levels of district- and state-level racial compositions. In the first row, I present the results of Models 1 and 3, where only racial characteristics are included. The second row plots the models that include covariates. Once again, all other covariates are held at their means.

\begin{figure}[!ht]

{\centering \includegraphics{bad_bills_files/figure-latex/mefs-cha-1} 

}

\caption{\label{fig:mef-dis}Provisions Sponsored by Chamber}\label{fig:mefs-cha}
\end{figure}

Table \ref{tab:chamb} and Figure \ref{fig:mef-dis} show that the relationship between district-level demographics and restrictive voting sponsorship are highly moderated by state-level characteristics, as expected. Namely, representatives from the whitest legislative districts in the most racially diverse states were by far the most likely to sponsor restrictive legislation. There is little relationship between district-level demographics and the probability of sponsoring a restrictive voting bill in the whitest states; it is only in the least-white states that this relationship becomes apparent. Although these relationships are moderated slightly with the inclusion of relevent sociodemographic and political covariates, the patterns remain clear: representatives from white districts in states with large nonwhite populations---whose (growing) demographic and political power could inspire fear in the more homogeneously-white parts of the state---were disproportionately likely to sponsor bills with provisions making voting more difficult. While Table \ref{tab:chamb} also indicates that partisanship plays a role (districts that voted in larger numbers for Biden were less likely to be represented by a lawmaker who sponsored a restrictive bill), the interplay between local and state demographics remains salient even after controlling for this.

\hypertarget{discussion}{%
\section{Discussion}\label{discussion}}

While legislators claim to pass restrictive voter policies under the guise of improved security, the data unambiguously points to white backlash to perceived racial threat. While discussing what she calls the ``political work of fraud allegations,'' Lorraine Minnite argued a decade ago that claims of fraud could be explained by ``the existence of marginalized subjects within the political culture whose presence alone stands in as the evidence of the alleged fraud'' (\protect\hyperlink{ref-Minnite2010}{Minnite 2010, 87}). In other words, the rhetoric of voter fraud draws boundaries around who does and does not count as a citizen, and whose political participation is inherently suspect.

A similar pattern emerges in this study of regressive voting laws introduced and passed in 2021, following the hard-fought and deeply polarizing 2020 presidential election and subsequent violent attempt to overturn the results on January 6th, 2021. Although the push to restrict voting access was extremely broad in 2021---more than 1 in 3 Americans lived in a district represented by a legislator who sponsored at least one of the 812 restrictive provisions---the push was concentrated in states and legislative districts with particular, systematic demographic characteristics.

\newpage

\hypertarget{references}{%
\section*{References}\label{references}}
\addcontentsline{toc}{section}{References}

\hypertarget{refs}{}
\begin{CSLReferences}{1}{0}
\leavevmode\hypertarget{ref-Andrews2015}{}%
Andrews, Kenneth T., and Charles Seguin. 2015. {``Group {Threat} and {Policy Change}: {The Spatial Dynamics} of {Prohibition Politics}, 1890{}.''} \emph{American Journal of Sociology} 121 (2): 475--510. \url{https://doi.org/10.1086/682134}.

\leavevmode\hypertarget{ref-BaconJr.2022}{}%
Bacon, Perry, Jr. 2022. {``Opinion \textbar{} {An} Anti-{Black} Backlash {} with No End in Sight.''} \emph{Washington Post}, January.

\leavevmode\hypertarget{ref-Beck2000}{}%
Beck, E. M. 2000. {``Guess {Who}'s {Coming} to {Town}: {White Supremacy}, {Ethnic Competition}, and {Social Change}.''} \emph{Sociological Focus} 33 (2): 153--74. \url{https://doi.org/10.1080/00380237.2000.10571163}.

\leavevmode\hypertarget{ref-Behrens2003}{}%
Behrens, Angela, Christopher Uggen, and Jeff Manza. 2003. {``Ballot {Manipulation} and the {`{Menace} of {Negro Domination}'}: {Racial Threat} and {Felon Disenfranchisement} in the {United States}, 1850{}.''} \emph{American Journal of Sociology} 109 (3): 559--605. \url{https://doi.org/10.1086/378647}.

\leavevmode\hypertarget{ref-Bentele2013}{}%
Bentele, Keith G., and Erin E. O'Brien. 2013. {``Jim {Crow} 2.0? {Why States Consider} and {Adopt Restrictive Voter Access Policies}.''} \emph{Perspectives on Politics} 11 (4): 1088--1116. \url{https://doi.org/10.1017/S1537592713002843}.

\leavevmode\hypertarget{ref-Biggers2015}{}%
Biggers, Daniel R., and Michael J. Hanmer. 2015. {``Who {Makes Voting Convenient}? {Explaining} the {Adoption} of {Early} and {No-Excuse Absentee Voting} in the {American States}.''} \emph{State Politics \& Policy Quarterly} 15 (2): 192--210.

\leavevmode\hypertarget{ref-Blumer1958}{}%
Blumer, Herbert. 1958. {``Race {Prejudice} as a {Sense} of {Group Position}.''} \emph{The Pacific Sociological Review} 1 (1): 3--7. \url{https://doi.org/10.2307/1388607}.

\leavevmode\hypertarget{ref-Cuthbert2022}{}%
Cuthbert, Lane, and Alexander Theodoridis. 2022. {``Analysis \textbar{} {Do Republicans} Really Believe {Trump} Won the 2020 Election? {Our} Research Suggests That They Do.''} \emph{Washington Post}, January.

\leavevmode\hypertarget{ref-Dale2020}{}%
Dale, Daniel. 2020. {``Fact Checking {Trump}'s Barrage of Lies over the Weekend.''} \emph{CNN}, November.

\leavevmode\hypertarget{ref-Eubank2022}{}%
Eubank, Nicholas, and Adriane Fresh. 2022. {``Enfranchisement and {Incarceration} After the 1965 {Voting Rights Act}.''} \emph{American Political Science Review}, January, 1--16. \url{https://doi.org/10.1017/S0003055421001337}.

\leavevmode\hypertarget{ref-Hicks2015}{}%
Hicks, William D., Seth C. McKee, Mitchell D. Sellers, and Daniel A. Smith. 2015. {``A {Principle} or a {Strategy}? {Voter Identification Laws} and {Partisan Competition} in the {American States}.''} \emph{Political Research Quarterly} 68 (1): 18--33. \url{https://doi.org/10.1177/1065912914554039}.

\leavevmode\hypertarget{ref-Juelich2020}{}%
Juelich, Courtney L., and Joseph A. Coll. 2020. {``Rock the {Vote} or {Block} the {Vote}? {How} the {Cost} of {Voting Affects} the {Voting Behavior} of {American Youth}: {Part} of {Special Symposium} on {Election Sciences}.''} \emph{American Politics Research} 48 (6): 719--24. \url{https://doi.org/10.1177/1532673X20920265}.

\leavevmode\hypertarget{ref-Kennedy2021}{}%
Kennedy, John, and Zac Anderson. 2021. {``Records Show Efforts to Gain Partisan Advantage Through Election Law Billed as Security Measure.''} \emph{Sarasota Herald-Tribune}, September.

\leavevmode\hypertarget{ref-Knowles2021}{}%
Knowles, Hannah. 2021. {``A {Texas} Bill Drew Ire for Saying It Would Preserve {`Purity of the Ballot Box.'} {Here}'s the Phrase's History.''} \emph{Washington Post}, May.

\leavevmode\hypertarget{ref-May2022}{}%
May, Payton. 2022. {``State Lawmakers Debate: {Is} Election Fraud a Problem in {Oklahoma}?''} \emph{KOKH}, March.

\leavevmode\hypertarget{ref-Minnite2010}{}%
Minnite, Lorraine C. 2010. \emph{The {Myth} of {Voter Fraud}}. {Ithaca, UNITED STATES}: {Cornell University Press}.

\leavevmode\hypertarget{ref-MITElectionDataandScienceLab2021}{}%
MIT Election Data and Science Lab. 2021. {``U.{S}. {President} 1976--2020.''} {Harvard Dataverse}. \url{https://doi.org/10.7910/DVN/42MVDX}.

\leavevmode\hypertarget{ref-Morris2021b}{}%
Morris, Kevin, and Myrna Pérez. 2021. {``Barriers to the {Ballot Box} in {Texas}.''} In \emph{Mexican {American Civil Rights} in {Texas}}, edited by Robert Brischetto and J. Richard Avena, 225--41. {Michigan State University Press}. \url{https://doi.org/10.14321/j.ctv1wsgrvs}.

\leavevmode\hypertarget{ref-Pabayo2021}{}%
Pabayo, Roman, Sze Yan Liu, Erin Grinshteyn, Daniel M. Cook, and Peter Muennig. 2021. {``Barriers to {Voting} and {Access} to {Health Insurance Among US Adults}: {A Cross-Sectional Study}.''} \emph{The Lancet Regional Health - Americas} 2 (October): 100026. \url{https://doi.org/10.1016/j.lana.2021.100026}.

\leavevmode\hypertarget{ref-Piven2009}{}%
Piven, Frances Fox, Lorraine Carol Minnite, and Margaret Groarke. 2009. \emph{Keeping down the Black Vote: Race and the Demobilization of {American} Voters}. {New York}: {New Press}.

\leavevmode\hypertarget{ref-Rackey2022}{}%
Rackey, John D., and C. Tyler Godines Camarillo. 2022. {``Stress {Test}: {Three Case Studies} on {Vote} by {Mail During} a {Global Pandemic}.''} In \emph{The {Roads} to {Congress} 2020: {Campaigning} in the {Era} of {Trump} and {COVID-19}}, edited by Sean D. Foreman, Marcia L. Godwin, and Walter Clark Wilson, 37--52. {Cham}: {Springer International Publishing}. \url{https://doi.org/10.1007/978-3-030-82521-8_3}.

\leavevmode\hypertarget{ref-Rutenberg2020}{}%
Rutenberg, Jim, Nick Corasaniti, and Alan Feuer. 2020. {``Trump's {Fraud Claims Died} in {Court}, but the {Myth} of {Stolen Elections Lives On}.''} \emph{The New York Times}, December.

\leavevmode\hypertarget{ref-Schraufnagel2020}{}%
Schraufnagel, Scot, Michael J. Pomante II, and Quan Li. 2020. {``Cost of {Voting} in the {American States}: 2020.''} \emph{Election Law Journal: Rules, Politics, and Policy} 19 (4): 503--9. \url{https://doi.org/10.1089/elj.2020.0666}.

\leavevmode\hypertarget{ref-Tilly1978}{}%
Tilly, Charles. 1978. \emph{From Mobilization to Revolution}. {New York}: {Random house}.

\leavevmode\hypertarget{ref-Tilly2015}{}%
Tilly, Charles, and Sidney Tarrow. 2015. \emph{Contentious {Politics}}. {New York, UNITED STATES}: {Oxford University Press, Incorporated}.

\leavevmode\hypertarget{ref-Uggen2020}{}%
Uggen, Christopher, Ryan Larson, Sarah Shannon, and Arleth Pulido-Nava. 2020. {``Locked {Out} 2020: {Estimates} of {People Denied Voting Rights Due} to a {Felony Conviction}.''} Research Report. {The Sentencing Project}.

\leavevmode\hypertarget{ref-Ura2021a}{}%
Ura, Alexa. 2021a. {``Texas {GOP}'s Voting Restrictions Bill Could Be Rewritten Behind Closed Doors After Final {House} Passage.''} \emph{The Texas Tribune}, May.

\leavevmode\hypertarget{ref-Ura2021b}{}%
---------. 2021b. {``Gov. {Greg Abbott} Signs {Texas} Voting Bill into Law, Overcoming {Democratic} Quorum Breaks.''} \emph{The Texas Tribune}, September.

\leavevmode\hypertarget{ref-Ura2021}{}%
Ura, Alexa, Chris Essig, and Madison Dong. 2021. {``Polling Places for Urban Voters of Color Would Be Cut Under {Texas Senate}'s Version of Voting Bill Being Negotiated with {House}.''} \emph{The Texas Tribune}, May.

\leavevmode\hypertarget{ref-VanDyke2002}{}%
Van Dyke, Nella, and Sarah A. Soule. 2002. {``Structural {Social Change} and the {Mobilizing Effect} of {Threat}: {Explaining Levels} of {Patriot} and {Militia Organizing} in the {United States}.''} \emph{Social Problems} 49 (4): 497--520. \url{https://doi.org/10.1525/sp.2002.49.4.497}.

\leavevmode\hypertarget{ref-VotingandElectionScienceTeam2022}{}%
Voting and Election Science Team. 2022. {``2020 {Precinct-Level Election Results}.''} {Harvard Dataverse}. \url{https://doi.org/10.7910/DVN/K7760H}.

\leavevmode\hypertarget{ref-Wang2012}{}%
Wang, Tova Andrea. 2012. \emph{The {Politics} of {Voter Suppression}: {Defending} and {Expanding Americans}' {Right} to {Vote}}. {Cornell University Press}. \url{https://doi.org/10.7591/cornell/9780801450853.001.0001}.

\leavevmode\hypertarget{ref-Yourish2021}{}%
Yourish, Karen, Larry Buchanan, and Denise Lu. 2021. {``The 147 {Republicans Who Voted} to {Overturn Election Results}.''} \emph{The New York Times}, January.

\leavevmode\hypertarget{ref-Zhang2018}{}%
Zhang, Yang, and Dingxin Zhao. 2018. {``The {Ecological} and {Spatial Contexts} of {Social Movements}.''} In \emph{The {Wiley Blackwell Companion} to {Social Movements}}, edited by David A. Snow, Sarah A. Soule, Hanspeter Kriesi, and Holly J. McCammon, 98--114. {John Wiley \& Sons, Ltd}. \url{https://doi.org/10.1002/9781119168577.ch5}.

\end{CSLReferences}

\end{document}
