% Options for packages loaded elsewhere
\PassOptionsToPackage{unicode}{hyperref}
\PassOptionsToPackage{hyphens}{url}
%
\documentclass[
  12pt,
]{article}
\usepackage{amsmath,amssymb}
\usepackage{lmodern}
\usepackage{ifxetex,ifluatex}
\ifnum 0\ifxetex 1\fi\ifluatex 1\fi=0 % if pdftex
  \usepackage[T1]{fontenc}
  \usepackage[utf8]{inputenc}
  \usepackage{textcomp} % provide euro and other symbols
\else % if luatex or xetex
  \usepackage{unicode-math}
  \defaultfontfeatures{Scale=MatchLowercase}
  \defaultfontfeatures[\rmfamily]{Ligatures=TeX,Scale=1}
\fi
% Use upquote if available, for straight quotes in verbatim environments
\IfFileExists{upquote.sty}{\usepackage{upquote}}{}
\IfFileExists{microtype.sty}{% use microtype if available
  \usepackage[]{microtype}
  \UseMicrotypeSet[protrusion]{basicmath} % disable protrusion for tt fonts
}{}
\makeatletter
\@ifundefined{KOMAClassName}{% if non-KOMA class
  \IfFileExists{parskip.sty}{%
    \usepackage{parskip}
  }{% else
    \setlength{\parindent}{0pt}
    \setlength{\parskip}{6pt plus 2pt minus 1pt}}
}{% if KOMA class
  \KOMAoptions{parskip=half}}
\makeatother
\usepackage{xcolor}
\IfFileExists{xurl.sty}{\usepackage{xurl}}{} % add URL line breaks if available
\IfFileExists{bookmark.sty}{\usepackage{bookmark}}{\usepackage{hyperref}}
\hypersetup{
  pdftitle={A Right Restricted},
  pdfauthor={Kevin Morris},
  hidelinks,
  pdfcreator={LaTeX via pandoc}}
\urlstyle{same} % disable monospaced font for URLs
\usepackage[margin=1in]{geometry}
\usepackage{longtable,booktabs,array}
\usepackage{calc} % for calculating minipage widths
% Correct order of tables after \paragraph or \subparagraph
\usepackage{etoolbox}
\makeatletter
\patchcmd\longtable{\par}{\if@noskipsec\mbox{}\fi\par}{}{}
\makeatother
% Allow footnotes in longtable head/foot
\IfFileExists{footnotehyper.sty}{\usepackage{footnotehyper}}{\usepackage{footnote}}
\makesavenoteenv{longtable}
\usepackage{graphicx}
\makeatletter
\def\maxwidth{\ifdim\Gin@nat@width>\linewidth\linewidth\else\Gin@nat@width\fi}
\def\maxheight{\ifdim\Gin@nat@height>\textheight\textheight\else\Gin@nat@height\fi}
\makeatother
% Scale images if necessary, so that they will not overflow the page
% margins by default, and it is still possible to overwrite the defaults
% using explicit options in \includegraphics[width, height, ...]{}
\setkeys{Gin}{width=\maxwidth,height=\maxheight,keepaspectratio}
% Set default figure placement to htbp
\makeatletter
\def\fps@figure{htbp}
\makeatother
\setlength{\emergencystretch}{3em} % prevent overfull lines
\providecommand{\tightlist}{%
  \setlength{\itemsep}{0pt}\setlength{\parskip}{0pt}}
\setcounter{secnumdepth}{5}
\usepackage{rotating}
\usepackage{setspace}
\usepackage{lscape}
\usepackage{pdfpages}
\newcommand{\beginsupplement}{\setcounter{table}{0}  \renewcommand{\thetable}{A\arabic{table}} \setcounter{figure}{0} \renewcommand{\thefigure}{A\arabic{figure}}}
\usepackage{lineno}
\usepackage{booktabs}
\usepackage{longtable}
\usepackage{array}
\usepackage{multirow}
\usepackage{wrapfig}
\usepackage{float}
\usepackage{colortbl}
\usepackage{pdflscape}
\usepackage{tabu}
\usepackage{threeparttable}
\usepackage{threeparttablex}
\usepackage[normalem]{ulem}
\usepackage{makecell}
\usepackage{xcolor}
\ifluatex
  \usepackage{selnolig}  % disable illegal ligatures
\fi
\newlength{\cslhangindent}
\setlength{\cslhangindent}{1.5em}
\newlength{\csllabelwidth}
\setlength{\csllabelwidth}{3em}
\newenvironment{CSLReferences}[2] % #1 hanging-ident, #2 entry spacing
 {% don't indent paragraphs
  \setlength{\parindent}{0pt}
  % turn on hanging indent if param 1 is 1
  \ifodd #1 \everypar{\setlength{\hangindent}{\cslhangindent}}\ignorespaces\fi
  % set entry spacing
  \ifnum #2 > 0
  \setlength{\parskip}{#2\baselineskip}
  \fi
 }%
 {}
\usepackage{calc}
\newcommand{\CSLBlock}[1]{#1\hfill\break}
\newcommand{\CSLLeftMargin}[1]{\parbox[t]{\csllabelwidth}{#1}}
\newcommand{\CSLRightInline}[1]{\parbox[t]{\linewidth - \csllabelwidth}{#1}\break}
\newcommand{\CSLIndent}[1]{\hspace{\cslhangindent}#1}

\title{A Right Restricted\thanks{TKTKT}}
\usepackage{etoolbox}
\makeatletter
\providecommand{\subtitle}[1]{% add subtitle to \maketitle
  \apptocmd{\@title}{\par {\large #1 \par}}{}{}
}
\makeatother
\subtitle{Understanding the Introduction and Passage of Restrictive Voting Laws}
\author{Kevin Morris\footnote{Researcher, Brennan Center for Justice (kevin.morris{[}at{]}nyu.edu).}}
\date{March 07, 2022}

\begin{document}
\maketitle
\begin{abstract}
TKTKT
\end{abstract}

\pagenumbering{gobble}
\pagebreak

\pagenumbering{arabic}
\doublespacing

\hypertarget{introduction}{%
\section{Introduction}\label{introduction}}

On May 7, 2021, Texas legislators in the state's House of Representatives debated and passed Senate Bill 7, an omnibus bill restricting voting in various ways. The bill would reduce access to mail voting, ban drive-through and 24-hour voting, and require large counties to redistribute their polling places away from Black and Latino neighborhoods (\protect\hyperlink{ref-Ura2021a}{Ura 2021a}; \protect\hyperlink{ref-Ura2021}{Ura, Essig, and Dong 2021}). Although this particular bill would ultimately fail after Democratic legislators broke quorum before the final vote by leaving the state, many of these provisions would ultimately become law as part of Senate Bill 1 during a special session called by the governor (\protect\hyperlink{ref-Ura2021b}{Ura 2021b}).

The debate in the House, however, was marked by an argument about a single phrase, used in the opening text of the bill. Senate Bill 7's self-described purpose was to ``detect and punish fraud and preserve the purity of the ballot box'' (§1.02). This phrase---``the purity of the ballot box''---has a long history in Texas, enshrined in the state's 1876 Constitution and used to defend the state's white primary that effectively shut nonwhites out of the political process for decades (\protect\hyperlink{ref-Knowles2021}{Knowles 2021}; \protect\hyperlink{ref-Morris2021b}{Morris and Pérez 2021}). Democratic representative Rafael Anchía questioned the bill author's use of this ``specific set of words that has a lot of meaning in state history,'' (quoted in \protect\hyperlink{ref-Knowles2021}{Knowles 2021}) saying the constitutional provision ``was drafted specifically to disenfranchise Black people.'' The implication was clear: Texas legislators in 2021 were tapping into long-standing legal racism to pass new legislation that would disproportionately impact voters of color. The phrase was dropped from the final version of Senate Bill 1, passed in August.

The twin features of the introduction to Texas' Senate Bill 7---protection against fraud and appeals to purity---typified the 2021 legislative session around the country. After losing his re-election bid in November, then-president Donald Trump claimed repeatedly that the election had been stolen (\protect\hyperlink{ref-Dale2020}{Dale 2020}), a claim he has continued to maintain and that some 70\% of registered Republicans believed by early 2022 (\protect\hyperlink{ref-Cuthbert2022}{Cuthbert and Theodoridis 2022}). Many state legislators also justified their support for restrictive legislation in terms similar to Oklahoma State Representative Sean Roberts (sponsor of the restrictive restrictive HB 2842 and HB 2847), who told reporters that ``{[}I{]}t was very clear that the election was stolen from President Trump. We must do everything we can to close those loopholes'' (quoted in \protect\hyperlink{ref-May2022}{May 2022}). Concerns about election security were not limited to state legislators: 147 Congressional Republicans voted against the certification of the 2020 presidential election (\protect\hyperlink{ref-Yourish2021}{Yourish, Buchanan, and Lu 2021}).

Despite these widespread beliefs, no evidence of fraud arose following the 2020 election, calling into question the veracity of these stated reasons for supporting the restrictive legislation. As the New York Times explained: ``After bringing some 60 lawsuits, and even offering financial incentive for information about fraud, Mr.~Trump and his allies have failed to prove definitively any case of illegal voting on behalf of their opponent in court---not a single case of an undocumented immigrant casting a ballot, a citizen double voting, nor any credible evidence that legions of the voting dead gave Mr.~Biden a victory that wasn't his'' (\protect\hyperlink{ref-Rutenberg2020}{Rutenberg, Corasaniti, and Feuer 2020}). The lack of evidence of fraud to insinuate that this restrictive legislation is driven by racial animus (e.g., \protect\hyperlink{ref-BaconJr.2022}{Bacon 2022}).

This project relies on a comprehensive survey of voting-related bills introduced around the country in 2021 systematically collected by the Brennan Center for Justice to adjudicate between these competing explanations for the introduction and passage of restrictive voting legislation.\footnote{See \url{https://www.brennancenter.org/our-work/research-reports/voting-laws-roundup-december-2021}. Provision-level data available from the Brennan Center upon request.} I start by examining these bills in light of already-existing voting regimes. If concerns about fraudulent voting have purchase, I expect to find more restrictive bills introduced and passed in states with more permissive voting laws---and therefore, perhaps, greater opportunity for fraud.

I also consider whether these restrictive laws can be explained by partisanship. Republicans may be passing bills expected to benefit their electoral prospects, as some have argued (e.g., \protect\hyperlink{ref-Kennedy2021}{Kennedy and Anderson 2021}). A partisan explanation would find more restrictive provisions in competitive states controlled by Republicans looking to cement a perhaps-tenuous advantage (\protect\hyperlink{ref-Hicks2015}{Hicks et al. 2015}).

Finally, I interrogate whether legislative behavior related to voting bills can be explained by white backlash and racial threat as many observers have claimed. If these restrictive laws are a response to rising non-white political power, I expect to find their introduction and passage concentrated in states with large non-white populations, and support for them concentrated in the whitest legislative districts in racially diverse states (\protect\hyperlink{ref-Andrews2015}{Andrews and Seguin 2015}).

The results are unequivocal: states with large nonwhite populations where voting was \emph{already} restricted made it even more difficult to vote in 2021. Restrictive legislation was not introduced or passed more in states with more lax voting regimes, and President Biden's 2020 voteshare was not related to state-level legislative behavior. Moreover, it was representatives from the whitest districts in the least-white states that were the most likely to sponsor restrictive legislation. While partisanship does explain legislative sponsorship---representatives of districts with higher Trump voteshare were the most likely to sponsor restrictive legislation---these strong racial patterns remain after accounting for polarized voting. In short, theories of white backlash and racial threat explain the introduction and passage of regressive voting laws in 2021 far better than do theories of partisanship or fears of fraud founded in actual voting laws.

\hypertarget{recent-work-on-restrictive-voting-laws}{%
\section{Recent Work on Restrictive voting Laws}\label{recent-work-on-restrictive-voting-laws}}

Over the past 15 years, scholars have explored the introduction and passage of restrictive of restrictive voting laws across the country. This work has largely focused on state-level factors, with a general consensus that these laws find the most fertile ground in states with large demographic change and a growing nonwhite electorate (\protect\hyperlink{ref-Bentele2013}{Bentele and O'Brien 2013}), where large numbers of Black Americans reside (\protect\hyperlink{ref-Behrens2003}{Behrens, Uggen, and Manza 2003}), and in electorally competitive states where Republicans hold a slight edge (\protect\hyperlink{ref-Hicks2015}{Hicks et al. 2015}).

\protect\hyperlink{ref-Behrens2003}{Behrens, Uggen, and Manza} (\protect\hyperlink{ref-Behrens2003}{2003}) uses a historical approach to understand the passage of laws disenfranchising citizens convicted of felony offenses. As they note, all but two American states restrict voting rights for at least some incarcerated citizens; the two that do not---Maine and Vermont---are also the two whitest states in the nation. Behrens and colleagues document the rise of these restrictive laws in the aftermath of the passage of the 14th and 15th Amendments, expanding formal citizenship and granting voting rights to Black men. Drawing on \protect\hyperlink{ref-Blumer1958}{Blumer} (\protect\hyperlink{ref-Blumer1958}{1958}) and other scholars of group threat, they argue that white (male) Americans were threatened by the prospect that their sole control over the political domain was no longer so secure. Of course, their claims to racial political dominance were threatened proportionate to the number of nonwhite potential voters; as such, states with larger nonwhite populations had political incentives to develop new ways to disenfranchise Black men. They find strong support for the theory that the widespread adoption of felony disenfranchisement rules rose from this threat. ``Our key finding can be summarized concisely and forcefully,'' they write. ``The racial composition of state prisons is firmly associated with the adoption of state felon disenfranchisement laws. States with greater nonwhite prison populations have been more likely to ban convicted felons from voting than states with proportionally fewer nonwhites in the criminal justice system'' (\protect\hyperlink{ref-Behrens2003}{Behrens, Uggen, and Manza 2003, 596}). Their conclusions have been corroborated more recently. \protect\hyperlink{ref-Eubank2022}{Eubank and Fresh} (\protect\hyperlink{ref-Eubank2022}{2022}) finds that states subject to strict federal oversight under the 1965 Voting Rights Act's Section 5 selectively increased the incarceration of Black Americans, providing further evidence that increased political opportunity for racial minorities leads white majorities to seek other ways of restricting their effective power.

Of course, the incarceration of citizens and subsequent legal disenfranchisement is perhaps only the most drastic example of curtailing access to the ballot.\footnote{It bears noting, however, that being drastic does not mean it is uncommon. More than 6\% of Black Americans were legally disenfranchised in 2020 due to a felony conviction. This number topped 10\% in 7 of the 33 states where the Black voting age population exceeded 100,000 (\protect\hyperlink{ref-Uggen2020}{Uggen et al. 2020}).} Might less extreme attempts to limit the pool of eligible voters follow a similar pattern? And do such considerations structure legislative behavior in the modern era? \protect\hyperlink{ref-Bentele2013}{Bentele and O'Brien} (\protect\hyperlink{ref-Bentele2013}{2013}) consider the introduction and passage of 5 types of restrictive voting legislation (``photo identification requirements, proof of citizenship requirements, laws that introduce restrictions on voter registration, restrictions on absentee and early voting, and restrictions on participation by felons'' (1095)) over the 2006--2011 period. They conclude that the strongest predictor of the introduction and passage of restrictive voting laws is the political power demonstrated by racial and ethnic minorities, arguing that ``legislative developments in this policy area remain heavily shaped by racial considerations'' (\protect\hyperlink{ref-Bentele2013}{Bentele and O'Brien 2013, 1089}). At the same time, they find no evidence that prevalence of voter fraud impacted the introduction of restrictive provisions and that it is ``only a minor contributing factor'' to the passage of these laws in 2011 (1103).

A further insight from \protect\hyperlink{ref-Bentele2013}{Bentele and O'Brien} (\protect\hyperlink{ref-Bentele2013}{2013})---that restrictive provisions are passed most frequently in electorally-competitive states---is corroborated by \protect\hyperlink{ref-Hicks2015}{Hicks et al.} (\protect\hyperlink{ref-Hicks2015}{2015}). EXPAND

This scholarship sheds important light on where restrictive voting laws are the most likely to go into effect, and the results are not encouraging. There is strong evidence that racial threat predicts the passage of these restrictive bills across the country, even as legislators proclaim that the changes are needed to combat widespread fraud (see, for instance, \protect\hyperlink{ref-Piven2009}{Piven, Minnite, and Groarke 2009}; \protect\hyperlink{ref-Minnite2010}{Minnite 2010}). Important as this research has been, however, it fails to explain the full set of dynamics between demographic composition and bill introduction. The explosion in the introduction of restrictive voting laws in 2021 makes this clear: according to the data from the Brennan Center for Justice used throughout this project, just one state (Vermont) introduced \emph{no} voting bills in 2021 containing no restrictive provisions. Moreover, the number of restrictive provisions introduced and passed in 2021 has little historical precedent: 880 restrictive provisions were introduced and 93 were passed. By way of comparison, \protect\hyperlink{ref-Bentele2013}{Bentele and O'Brien} (\protect\hyperlink{ref-Bentele2013}{2013})---which also used data from the Brennan Center---calls roughly the roughly 20 passed provisions in 2011 a ``dramatic increase'' (1088; see their Figure 2).

Clearly, something more complex than state-level factors are at play in the contemporary push to restrict voting rights. By considering not only state-level factors but also examining the demographics of the districts represented by legislators who introduce, co-sponsor, and vote for these restrictive bills, this project marks a significant step forward in understanding how racial threat's influence on the policy-making process is mediated by factors at multiple political levels. The following section steps back to engage with the (racial) threat literature and, more specifically, consider how spatially-situated theories of threat help us to formulate expectations about the roles played by state and local factors in the introduction and passage of restrictive voting laws.

\hypertarget{a-changing-electorate-and-threat}{%
\section{A Changing Electorate and Threat}\label{a-changing-electorate-and-threat}}

Scholars across the social sciences have long noted the importance of threat to the policy-making process; indeed, each of the studies discussed in the previous section implicitly or explicitly draw on this literature. \protect\hyperlink{ref-Tilly1978}{Tilly} (\protect\hyperlink{ref-Tilly1978}{1978}) separates collective action into three categories: defensive, offensive, and preparatory (73). Of these, two---defensive and preparatory---are explicitly linked to threats, where political actors pool their resources to fend of challenges to their interests, or to regain what has already been lost. \protect\hyperlink{ref-Beck2000}{Beck} (\protect\hyperlink{ref-Beck2000}{2000}) extends this theory to note that defensive actors need only \emph{perceive} that their interests have been compromised to mobilize in a reactionary way; the \emph{reality} of any worsened station is perhaps less important. These threats can take multiple forms, be they economic, political, or demographic (\protect\hyperlink{ref-VanDyke2002}{Van Dyke and Soule 2002}).

The social sciences are replete with empirical evidence of the mobilizing effect of these three forms of threat. Early work on collective action demonstrates that unemployed workers

Only recently, however, have the spatial dynamics of racial threat and policy threat been considered with serious attention. Andrews and Seguin (\protect\hyperlink{ref-Andrews2015}{2015, 476}) argues that ``threat arises primarily from interactions between spatially proximate units at the local level\ldots{} and therefore higher-level policy change at the state level is not reducible to the variables driving local policy.'' In other words, examining local and state characteristics alone is not sufficient to understand legislator support for racially conservative policy changes; instead, responses to racial threat arise from the \emph{interaction} of these circumstances.

\hypertarget{methods-and-expectations}{%
\section{Methods and Expectations}\label{methods-and-expectations}}

Throughout our analyses, I rely on the Voting Laws Roundup, a project of the Brennan Center for Justice at NYU School of Law. The Brennan Center systematically reviews all laws introduced around the country that relate to voting and the administration of elections in each state. The Brennan Center identifies these bills using string-searches in Westlaw, and then separates each bill introduced into its constituent provisions, using two coders to designate each provision as ``restrictive,'' ``neutral,'' or ``expansive.'' Each provision is also assigned to a category describing its effect (categories include effects such as ``voter ID,'' ``polling place count,'' or ``funding for poll workers''). In 2021, the Brennan Center identified 2,859 bills that were introduced and related to voting, consisting of 5,927 provisions. Of these provisions, 812 were deemed restrictive; 3,445 neutral; and 1,670 expansive. A total of 72 restrictive provisions were passed into law that year; 157 neutral provisions and 19 expansive ones were also passed. Figure \ref{fig:cols} shows the categorical breakdown of restrictive provisions introduced and passed, while Figure \ref{fig:maps} shows the geographical distribution of these provisions.

\begin{figure}[!ht]

{\centering \includegraphics{bad_bills_files/figure-latex/cols-1} 

}

\caption{\label{fig:cols}Categories of Restrictive Provisions, 2021}\label{fig:cols}
\end{figure}

\begin{figure}[!ht]

{\centering \includegraphics{bad_bills_files/figure-latex/maps-1} 

}

\caption{\label{fig:maps}Restrictive Provisions, 2021}\label{fig:maps}
\end{figure}

These provisions are then merged with data from LegiScan, which tracks bills of all kinds in statehouses throughout the United States. The LegiScan data includes information on bill sponsorship, roll-call votes, and dates of introduction. These data are used to identify the (co)sponsors of all the voting laws identified by the Brennan Center, as well as the legislators voting in favor or against them.

To account for pre-2021 variation in the electoral landscape, I incorporate each state's score on the (pre-COVID) 2020 Cost of Voting Index (\protect\hyperlink{ref-Schraufnagel2020}{Schraufnagel, Pomante II, and Li 2020}) (COVI). This index captures many of the same items tracked by the Brennan Center, such as whether a state has same-day registration, voter ID laws, and the number of days of early voting. The index is widely used by social scientists to measure the difficulty of voting (e.g. \protect\hyperlink{ref-Juelich2020}{Juelich and Coll 2020}; \protect\hyperlink{ref-Pabayo2021}{Pabayo et al. 2021}; \protect\hyperlink{ref-Rackey2022}{Rackey and Camarillo 2022}). A higher value of the COVI indicates that voting is more difficult in that state, and the 2020 COVI ranges from -2.9 in Oregon to 1.4 in New Hampshire.

I also control for the partisan control of each state using two pieces of data. Following \protect\hyperlink{ref-Hicks2015}{Hicks et al.} (\protect\hyperlink{ref-Hicks2015}{2015}), it seems possible that electorally competitive states where Republicans hold unified power would be most likely to introduce and pass bad provisions. I thus include 2 dummies: one measuring whether the state was competitive (that is, Biden received between 45\% and 55\% of votes), and one measuring whether Republicans held unified control in 2021.

In addition to these measures of voting legislation and policy environment, I incorporate demographic data at the state and legislative-district level from the Census Bureau's American Communities Survey. Except where noted, these estimates are the 5-year numbers ending with 2019.\footnote{At the time of writing, the 2020 estimates were not yet available. The study will be updated when they become available.}

Based on the literature discussed above, I pose three hypotheses at the state-level. It is worth noting that \textbf{H1a} and \textbf{H1b} are mutually exclusive.

\begin{itemize}
\item
  \textbf{H1a}: Following stated concerns of legislators around the country, I expect that more restrictive voting provisions were introduced and passed in 2021 in states where the COVI was lower in 2020.
\item
  \textbf{H1b}: Following literature indicating that the pre-2021 difficulty of voting in a state was already indicative of how likely that state was to react to displays of nonwhite political power, I expect states with higher COVIs introduced and passed more restrictive legislation in 2021 than states with lower COVIs.
\item
  \textbf{H1c}: Following theories of partisan advantage, I expect that more restrictive voting provisions were introduced in electorally-competitive states where Republicans had unified control over state government.
\end{itemize}

If I find support for \textbf{H1b}, this implies that the COVI is, at least in part, a measure of state-level reactiveness to racial threat. This further implies that less-white states with a higher 2020 COVI were more likely to respond to the 2020 election with restrictive election legislation. As such, \textbf{H2}: conditional on H1b, I expect that less-white states with a higher 2020 COVI introduced and passed more restrictive provisions in 2021 than less-white states with a lower 2020 COVI.

After presenting tests of these hypotheses at the state-level, I move to analyses at the level of the legislative chamber. The empirical framework is the same, though the dependent variable changes from the \emph{count} of restrictive provisions to a dummy variable indicating \emph{whether} a legislator sponsored restrictive provisions, due to smaller scale.

In light of recent sociological work on the geography of racial threat (\protect\hyperlink{ref-Andrews2015}{Andrews and Seguin 2015}), \textbf{H3}: I expect that whiter districts in less-white states were the most likely to be represented by legislators that sponsored restrictive voting provisions.

\hypertarget{results}{%
\section{Results}\label{results}}

\hypertarget{state-level-legislative-behavior}{%
\subsection{State-Level Legislative Behavior}\label{state-level-legislative-behavior}}

In Table \ref{tab:state}, I present the results of tests on my first set of hypotheses. Models 1--3 present the results of models where the dependent variable is the number of \emph{introduced} restrictive provisions; Models 4--6, on the other hand, explore counts of passed provisions. By way of reminder, Hypotheses 1a and 1b are mutually exclusive; as such, they can be tested at the same time, and are included in Models 1, 3, 4, and 6.

Because race and partisan control might be related, in Models 1 and 4, I do not control for the partisan environment of each state. In these models, I control for a set of covariates but am primarily interested in the relationship between states' racial composition and pre-2021 COVI, and provision counts. Meanwhile, in Models 2 and 5, I drop the race and COVI variables, controlling only for partisan context (and the other control covariates). Finally, Models 3 and 6 incorporate the race and COVI covariates, along with those measuring partisan control.

These models uncover substantial support for the hypotheses arguing that race matters more than partisan control. The addition of partisan control variables to models controlling for race and COVI (that is, moving from Model 1 to Model 3, and Model 4 to Model 6) does not meaningfully alter the size of the cofficients on race and COVI (or their statistical significance levels), nor does their addition improve the R\textsuperscript{2} meaningfully (in fact, the adjusted R\textsuperscript{2} actually \emph{decreases} with the addition of partisan controls when looking at passed provisions).

The same is not true of adding race and COVI covariates to the partisan control models (moving from Model 2 to 3, and 5 to 6). The coefficients on \emph{Unified Republican Control} change dramatically, as do those on \emph{Competitive in 2020 × Unified Republican Control}. Moreover, the addition of these race and COVI variables improves the R\textsuperscript{2} by well over 50\%.
\newpage

\begin{landscape}
\input{"../temp/state_reg.tex"}
\end{landscape}

Because interaction coefficients can be difficult to interpret in table format, I here present the predicted probabilities of counts of introduced and passed provisions in Figures \ref{fig:mefs-state} and \ref{fig:mefs-state-comp}. The first row of each figure correspond to the model in Table \ref{tab:state} where the other set of key variables are excluded (i.e., Models 1 and 4 in Figure \ref{mefs-state} and Models 2 and 5 in Figure \ref{mefs-state-comp}). The second row, however, includes the full set of covariates in both figures, plotting the relationships from Models 3 and 6. All other covariates are held at their means.

\begin{figure}[!ht]

{\centering \includegraphics{bad_bills_files/figure-latex/mefs-state-1} 

}

\caption{\label{fig:mefs-state}Restrictive Provisions Introduced and Passed, by Race and 2020 COVI}\label{fig:mefs-state}
\end{figure}

\begin{figure}[!ht]

{\centering \includegraphics{bad_bills_files/figure-latex/mefs-state-comp-1} 

}

\caption{\label{fig:mefs-state-comf}Restrictive Provisions Introduced and Passed, by Competitiveness and Republican Control}\label{fig:mefs-state-comp}
\end{figure}

Figure \ref{fig:mefs-state} lays these relationships out starkly: less-white states where the COVI was already high were substantially more likely to introduce and pass restrictive legislation than A) whiter states with high COVIs, and B) less-white states with lower COVIs. Generally speaking, the whitest states did not consider restrictive legislation in 2021, regardless of their 2020 COVI. The same was true of less-white states with lower 2020 COVIs. Figure \ref{fig:mefs-state} indicates that the regressive legislation of 2021 was concentrated in a handful of states and provides very strong support for H1b and H2 (and, by extension, H1b). In the Supplementary Information, I show that these relationships are very different when we consider the introduction and passage of \emph{expansive} voting bills. It is worth noting that these curves are \emph{not meaningfully changed} in the second row of Figure \ref{fig:mefs-state} when I control for partisanship. In other words, the observed relationships cannot be explained by the fact that race and partisanship are closely tied.

As previous scholarship has detailed, base-rates in the difficulty of voting were probably reflective of a white majority feeling threatened by non-white political power. Put differently, the 2020 COVI probably picked up on salient differences among less-white states that were correlated with racial threat. The states with the higher COVIs in 2020, therefore, were more susceptible to feeling threatened by the results of the 2020 election, leading to racial backlash.

Figure \ref{mefs-state-comp}, on the other hand, undermines support for H1c. Here, we see that \emph{before} accounting for states' racial composition and 2020 COVI, there is a partisan story: states with unified Republican control introduced and passed substantially more restrictive provisions if they were competitive in the 2020 presidential election. After controlling for race and COVI, however, these relationships disappear. This indicates that, insofar as we can observe a partisan pattern to the introduction and passage of these bills, it is explained by the role that race plays in the construction of partisan control and presidential election competitiveness.

In short: theories of racial threat and white backlash do a far better job of explaining the observed patterns in the 2021 legislative context than do arguments about shoring up overly-permissive voting regimes, or those pointing only to partisan explanations.

\hypertarget{district-level-legislative-behavior}{%
\subsection{District-Level Legislative Behavior}\label{district-level-legislative-behavior}}

Having determined that the geographic concentration of the restrictive voting bills in 2021 was highly consistent with theories of racial threat, I turn to the sub-state level to better understand these processes. As discussed above, I expect to find evidence of backlash against racial threat in the whitest parts of the least white states.

Table \ref{tab:chamb} presents the results of OLS regressions investigating the likelihood that a legislator from a given district sponsored at least one restrictive (Models 1--4) or expansive (Models 5--8) voting provision in 2021.\footnote{The ACS is missing demographic data on a handful of small legislative districts, explaining the variability in the observation count. However, the share of the population living in upper and lower chambers without demographic data is just 0.5\% and 1.1\%, respectively.} About 21\% of lower chamber legislators and 20\% of upper chamber legislators sponsored restrictive provisions; these numbers are 33\% and 31\% for expansive provisions, respectively.

\newpage
\begin{landscape}
\input{"../temp/district_reg.tex"}
\end{landscape}

As before, I plot the predicted probability of bill sponsorship for provisions to aid in the interpretation of the substantive relationships. Figure \ref{fig:mef-dis} presents like predicted likelihood of the sponsorship of a provision at different levels of district- and state-level racial compositions. Once again, all other covariates are held at their means.

\begin{figure}[!ht]

{\centering \includegraphics{bad_bills_files/figure-latex/mefs-cha-1} 

}

\caption{\label{fig:mef-dis}Provisions Sponsored by Chamber}\label{fig:mefs-cha}
\end{figure}

Table \ref{tab:chamb} and Figure \ref{fig:mef-dis} make a number of things immediately clear. Firstly, it is apparent that in both upper and lower legislative chambers, legislators from whiter districts are \emph{more} likely to introduce voting bills with restrictive provisions and \emph{less} likely to introduce expansive voting bills, ceteris paribus. Although Table \ref{tab:chamb} indicates that this relationship between district-level demographics is moderated by the share of the state that is white, Figure \ref{fig:mef-dis}(B) demonstrates that the substantive relationship is really very similar: the slope of the line in less-white states is effectively the same as that in the whiter states.

Figure \ref{fig:mef-dis}(A), however, provides evidence that strongly supports H3. While legislators from less-white districts in less-white states are among the \emph{least} likely to sponsor restrictive bills, the legislators from the whitest parts of these same states are among the \emph{most} likely to do so.

\hypertarget{discussion}{%
\section{Discussion}\label{discussion}}

While legislators claim to pass restrictive voter policies under the guise of improved security, the data unambiguously points to white backlash to perceived racial threat. While discussing what she calls the ``political work of fraud allegations,'' Lorraine Minnite argued a decade ago that claims of fraud could be explained by ``the existence of marginalized subjects within the political culture whose presence alone stands in as the evidence of the alleged fraud'' (\protect\hyperlink{ref-Minnite2010}{Minnite 2010, 87}). In other words, the rhetoric of voter fraud draws boundaries around who does and does not count as a citizen, and whose political participation is inherently suspect.

A similar pattern emerges in this study of regressive voting laws introduced and passed in 2021, following the hard-fought and deeply polarizing 2020 presidential election and subsequent violent attempt to overturn the results on January 6th, 2021. Although the push to restrict voting access was extremely broad in 2021---more than 1 in 3 Americans lived in a district represented by a legislator who sponsored at least one of the 812 restrictive provisions---the push was concentrated in states and legislative districts with particular, systematic demographic characteristics.

\newpage

\hypertarget{references}{%
\section*{References}\label{references}}
\addcontentsline{toc}{section}{References}

\hypertarget{refs}{}
\begin{CSLReferences}{1}{0}
\leavevmode\hypertarget{ref-Andrews2015}{}%
Andrews, Kenneth T., and Charles Seguin. 2015. {``Group {Threat} and {Policy Change}: {The Spatial Dynamics} of {Prohibition Politics}, 1890{}.''} \emph{American Journal of Sociology} 121 (2): 475--510. \url{https://doi.org/10.1086/682134}.

\leavevmode\hypertarget{ref-BaconJr.2022}{}%
Bacon, Perry, Jr. 2022. {``Opinion \textbar{} {An} Anti-{Black} Backlash {} with No End in Sight.''} \emph{Washington Post}, January.

\leavevmode\hypertarget{ref-Beck2000}{}%
Beck, E. M. 2000. {``Guess {Who}'s {Coming} to {Town}: {White Supremacy}, {Ethnic Competition}, and {Social Change}.''} \emph{Sociological Focus} 33 (2): 153--74. \url{https://doi.org/10.1080/00380237.2000.10571163}.

\leavevmode\hypertarget{ref-Behrens2003}{}%
Behrens, Angela, Christopher Uggen, and Jeff Manza. 2003. {``Ballot {Manipulation} and the {`{Menace} of {Negro Domination}'}: {Racial Threat} and {Felon Disenfranchisement} in the {United States}, 1850{}.''} \emph{American Journal of Sociology} 109 (3): 559--605. \url{https://doi.org/10.1086/378647}.

\leavevmode\hypertarget{ref-Bentele2013}{}%
Bentele, Keith G., and Erin E. O'Brien. 2013. {``Jim {Crow} 2.0? {Why States Consider} and {Adopt Restrictive Voter Access Policies}.''} \emph{Perspectives on Politics} 11 (4): 1088--1116. \url{https://doi.org/10.1017/S1537592713002843}.

\leavevmode\hypertarget{ref-Blumer1958}{}%
Blumer, Herbert. 1958. {``Race {Prejudice} as a {Sense} of {Group Position}.''} \emph{The Pacific Sociological Review} 1 (1): 3--7. \url{https://doi.org/10.2307/1388607}.

\leavevmode\hypertarget{ref-Cuthbert2022}{}%
Cuthbert, Lane, and Alexander Theodoridis. 2022. {``Analysis \textbar{} {Do Republicans} Really Believe {Trump} Won the 2020 Election? {Our} Research Suggests That They Do.''} \emph{Washington Post}, January.

\leavevmode\hypertarget{ref-Dale2020}{}%
Dale, Daniel. 2020. {``Fact Checking {Trump}'s Barrage of Lies over the Weekend.''} \emph{CNN}, November.

\leavevmode\hypertarget{ref-Eubank2022}{}%
Eubank, Nicholas, and Adriane Fresh. 2022. {``Enfranchisement and {Incarceration} After the 1965 {Voting Rights Act}.''} \emph{American Political Science Review}, January, 1--16. \url{https://doi.org/10.1017/S0003055421001337}.

\leavevmode\hypertarget{ref-Hicks2015}{}%
Hicks, William D., Seth C. McKee, Mitchell D. Sellers, and Daniel A. Smith. 2015. {``A {Principle} or a {Strategy}? {Voter Identification Laws} and {Partisan Competition} in the {American States}.''} \emph{Political Research Quarterly} 68 (1): 18--33. \url{https://doi.org/10.1177/1065912914554039}.

\leavevmode\hypertarget{ref-Juelich2020}{}%
Juelich, Courtney L., and Joseph A. Coll. 2020. {``Rock the {Vote} or {Block} the {Vote}? {How} the {Cost} of {Voting Affects} the {Voting Behavior} of {American Youth}: {Part} of {Special Symposium} on {Election Sciences}.''} \emph{American Politics Research} 48 (6): 719--24. \url{https://doi.org/10.1177/1532673X20920265}.

\leavevmode\hypertarget{ref-Kennedy2021}{}%
Kennedy, John, and Zac Anderson. 2021. {``Records Show Efforts to Gain Partisan Advantage Through Election Law Billed as Security Measure.''} \emph{Sarasota Herald-Tribune}, September.

\leavevmode\hypertarget{ref-Knowles2021}{}%
Knowles, Hannah. 2021. {``A {Texas} Bill Drew Ire for Saying It Would Preserve {`Purity of the Ballot Box.'} {Here}'s the Phrase's History.''} \emph{Washington Post}, May.

\leavevmode\hypertarget{ref-May2022}{}%
May, Payton. 2022. {``State Lawmakers Debate: {Is} Election Fraud a Problem in {Oklahoma}?''} \emph{KOKH}, March.

\leavevmode\hypertarget{ref-Minnite2010}{}%
Minnite, Lorraine C. 2010. \emph{The {Myth} of {Voter Fraud}}. {Ithaca, UNITED STATES}: {Cornell University Press}.

\leavevmode\hypertarget{ref-Morris2021b}{}%
Morris, Kevin, and Myrna Pérez. 2021. {``Barriers to the {Ballot Box} in {Texas}.''} In \emph{Mexican {American Civil Rights} in {Texas}}, edited by Robert Brischetto and J. Richard Avena, 225--41. {Michigan State University Press}. \url{https://doi.org/10.14321/j.ctv1wsgrvs}.

\leavevmode\hypertarget{ref-Pabayo2021}{}%
Pabayo, Roman, Sze Yan Liu, Erin Grinshteyn, Daniel M. Cook, and Peter Muennig. 2021. {``Barriers to {Voting} and {Access} to {Health Insurance Among US Adults}: {A Cross-Sectional Study}.''} \emph{The Lancet Regional Health - Americas} 2 (October): 100026. \url{https://doi.org/10.1016/j.lana.2021.100026}.

\leavevmode\hypertarget{ref-Piven2009}{}%
Piven, Frances Fox, Lorraine Carol Minnite, and Margaret Groarke. 2009. \emph{Keeping down the Black Vote: Race and the Demobilization of {American} Voters}. {New York}: {New Press}.

\leavevmode\hypertarget{ref-Rackey2022}{}%
Rackey, John D., and C. Tyler Godines Camarillo. 2022. {``Stress {Test}: {Three Case Studies} on {Vote} by {Mail During} a {Global Pandemic}.''} In \emph{The {Roads} to {Congress} 2020: {Campaigning} in the {Era} of {Trump} and {COVID-19}}, edited by Sean D. Foreman, Marcia L. Godwin, and Walter Clark Wilson, 37--52. {Cham}: {Springer International Publishing}. \url{https://doi.org/10.1007/978-3-030-82521-8_3}.

\leavevmode\hypertarget{ref-Rutenberg2020}{}%
Rutenberg, Jim, Nick Corasaniti, and Alan Feuer. 2020. {``Trump's {Fraud Claims Died} in {Court}, but the {Myth} of {Stolen Elections Lives On}.''} \emph{The New York Times}, December.

\leavevmode\hypertarget{ref-Schraufnagel2020}{}%
Schraufnagel, Scot, Michael J. Pomante II, and Quan Li. 2020. {``Cost of {Voting} in the {American States}: 2020.''} \emph{Election Law Journal: Rules, Politics, and Policy} 19 (4): 503--9. \url{https://doi.org/10.1089/elj.2020.0666}.

\leavevmode\hypertarget{ref-Tilly1978}{}%
Tilly, Charles. 1978. \emph{From Mobilization to Revolution}. {New York}: {Random house}.

\leavevmode\hypertarget{ref-Uggen2020}{}%
Uggen, Christopher, Ryan Larson, Sarah Shannon, and Arleth Pulido-Nava. 2020. {``Locked {Out} 2020: {Estimates} of {People Denied Voting Rights Due} to a {Felony Conviction}.''} Research Report. {The Sentencing Project}.

\leavevmode\hypertarget{ref-Ura2021a}{}%
Ura, Alexa. 2021a. {``Texas {GOP}'s Voting Restrictions Bill Could Be Rewritten Behind Closed Doors After Final {House} Passage.''} \emph{The Texas Tribune}, May.

\leavevmode\hypertarget{ref-Ura2021b}{}%
---------. 2021b. {``Gov. {Greg Abbott} Signs {Texas} Voting Bill into Law, Overcoming {Democratic} Quorum Breaks.''} \emph{The Texas Tribune}, September.

\leavevmode\hypertarget{ref-Ura2021}{}%
Ura, Alexa, Chris Essig, and Madison Dong. 2021. {``Polling Places for Urban Voters of Color Would Be Cut Under {Texas Senate}'s Version of Voting Bill Being Negotiated with {House}.''} \emph{The Texas Tribune}, May.

\leavevmode\hypertarget{ref-VanDyke2002}{}%
Van Dyke, Nella, and Sarah A. Soule. 2002. {``Structural {Social Change} and the {Mobilizing Effect} of {Threat}: {Explaining Levels} of {Patriot} and {Militia Organizing} in the {United States}.''} \emph{Social Problems} 49 (4): 497--520. \url{https://doi.org/10.1525/sp.2002.49.4.497}.

\leavevmode\hypertarget{ref-Yourish2021}{}%
Yourish, Karen, Larry Buchanan, and Denise Lu. 2021. {``The 147 {Republicans Who Voted} to {Overturn Election Results}.''} \emph{The New York Times}, January.

\end{CSLReferences}

\end{document}
